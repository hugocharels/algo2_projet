\documentclass {article}

% \usepackage{listings}

\usepackage{minted, xcolor}
% \usepackage{minted}
\definecolor{bg}{HTML}{282828}



\begin {document}
% \begintitlepage
% \centering
\author {Hugo Charels 000544051 \and Mickael Kovel 000396950}
\date {28 Avril 2023}
\title {Rapport de projet : Rectangle Tree}
\maketitle
\newpage


\tableofcontents
\newpage
\section {Introduction}


% Les polygones jouent un rôle essentiel dans de nombreuses applications, 
% qu'il s'agisse de représenter des régions administratives sur une carte, des informations géologiques, 
% des zones dans des dessins vectoriels ou même des imageries médicales. Lors de la manipulation de telles données, 
% une question courante se pose : à quel polygone appartient un point donné, tel qu'un clic de souris ou 
% la position d'un capteur GPS ? Cette question est résolue par le problème du "Point in Polygon" (PIP), 
% dont l'algorithme a été connu depuis 1962.
% \\
Dans de nombreuses applications, la manipulation de polygones est essentielle. Ces polygones peuvent représenter 
des régions administratives, des informations géologiques, des zones de dessins vectoriels ou encore des 
images médicales. Lorsqu'il s'agit de déterminer à quel polygone appartient un point donné, une question se pose : 
comment résoudre efficacement le problème du "Point in Polygon" (PIP) ?

% \\
L'algorithme PIP, connu depuis 1962, consiste à compter le nombre de fois qu'une demi-droite partant du point 
traverse une arête du polygone. Si ce nombre est pair, le point est à l'extérieur du polygone, sinon, 
il est contenu à l'intérieur. Cependant, si l'on a des milliers de polygones complexes à tester avec un 
grand nombre de points, une méthode naïve devient inefficace.

Une première optimisation consiste à utiliser l'enveloppe ou "minimum bounding rectangle" (MBR), 
qui est le plus petit rectangle horizontal englobant totalement le polygone. Avant d'appliquer l'algorithme PIP, 
on vérifie d'abord l'inclusion du point dans le MBR, ce qui permet de réduire le nombre d'appels à PIP.

Cependant, lorsque le nombre de polygones et de MBR est élevé, une approche hiérarchique basée sur l'algorithme R-Tree 
devient pertinente. En regroupant les MBR de manière optimale, on peut éviter de considérer tous les MBR pour 
chaque point à tester.

Dans ce projet, nous allons implémenter et évaluer deux variantes de l'algorithme R-Tree (quadratique et linéaire) 
pour résoudre efficacement le problème du PIP. Nous testerons ces variantes avec différents paramètres pour 
évaluer leurs performances.

Le rapport détaillera la méthodologie utilisée, les résultats obtenus et une analyse approfondie des performances 
des variantes de l'algorithme R-Tree. Nous discuterons également des défis rencontrés et des perspectives pour de 
futures recherches.


En conclusion, ce projet vise à résoudre efficacement le problème du PIP en utilisant l'algorithme 
R-Tree. 
Les résultats obtenus nous permettront de déterminer la meilleure variante en fonction des 
paramètres et de formuler des recommandations pour d'éventuelles améliorations.

\section {Structure}

Dans cette section, nous allons explorer la structure d'un R-Tree, une arborescence utilisée 
pour organiser efficacement les données géospatiales et résoudre le problème du 
"Point in Polygon" (PIP). L'objectif est de comprendre comment cette structure hiérarchique 
permet de regrouper les MBR (minimum bounding rectangles) de manière optimale.

Un R-Tree est une structure de données arborescente utilisée pour l'indexation spatiale des 
objets géométriques. Il est conçu pour stocker et organiser efficacement des données 
multidimensionnelles, telles que des polygones, des points ou des rectangles.

L'idée fondamentale d'un R-Tree est de regrouper les MBR de manière hiérarchique. 
Chaque nœud de l'arbre représente un MBR, qui peut être un polygone ou un rectangle englobant 
un groupe de MBR plus petits.
\begin{itemize}

    \item Nœuds internes : Les nœuds internes de l'arbre contiennent des MBR qui englobent 
	plusieurs autres MBR. 
	Ils servent de guides pour naviguer dans la structure et réduire la recherche de MBR 
	pertinents lors de la résolution du problème du PIP.

    \item Feuilles : Les nœuds feuilles de l'arbre contiennent les MBR individuels et sont 
	généralement associés à des objets géométriques spécifiques. 
	Ils sont utilisés pour effectuer les tests d'inclusion du point lors de	la résolution 
	du PIP.
\end{itemize}

L'algorithme R-Tree utilise des critères spécifiques pour regrouper les MBR de manière optimale. 
Ces critères visent à maximiser la compacité des groupes de MBR et minimiser leur superposition 
avec d'autres groupes.

\begin{itemize}
    \item Critère de regroupement spatial : Les MBR ayant une proximité spatiale élevée sont 
	regroupés ensemble pour réduire la recherche dans les parties de l'arbre non pertinentes 
	pour le point en cours d'évaluation.

    \item Critère de compacité : Lorsque plusieurs MBR sont candidats pour un regroupement, 
	celui qui crée le MBR englobant le plus compact est préféré. 
	Cela garantit une meilleure utilisation de l'espace de stockage et une réduction des 
	opérations de recherche.

\end{itemize}
Nous détaillerons la recherche d'un point dans un R-Tree dans la section \ref{recherche}.

\section {Création}

Dans cette section, nous allons aborder la création d'un R-Tree, en expliquant les étapes de base 
pour construire cette structure hiérarchique. 
Nous présenterons également deux algorithmes de construction : 
le split quadratique et le split linéaire.

La création d'un R-Tree implique de regrouper les MBR de manière hiérarchique. 
Pour cela, nous commençons avec un ensemble de MBR individuels et suivons un processus 
de division et de regroupement pour former l'arbre.

L'algorithme de construction de base utilise deux étapes principales : le split et le picknext. 
Le split consiste à diviser un groupe de MBR en deux groupes plus petits, 
tandis que le picknext sélectionne le prochain MBR à insérer dans l'arbre.

Il existe deux variantes d'algorithmes de split : le split quadratique et le split linéaire. 
Explorons-les plus en détail.


\subsection {Split quadratique}
Le split quadratique est un algorithme de division utilisé dans la construction d'un R-Tree. 
Il se base sur le concept de compacité des MBR pour choisir les MBR à diviser.

% L'idée du split quadratique est de sélectionner deux MBR initiaux qui englobent le plus grand espace possible. 
% Pour cela, nous calculons toutes les paires possibles de MBR et choisissons celle qui crée les deux MBR les plus 
% compacts.
L'idée du split quadratique est de sélectionner deux MBR initiaux qui sont le plus éloignés possible.
Pour cela, nous calculons toutes les paires possibles de MBR et choisissons celle qui dont 
l'aire d'encadrement est la plus grande.

Avantages du split quadratique :

\begin{itemize}
    \item Il tend à créer une structure plus équilibrée en termes de taille des nœuds et de 
	compacité des MBR.
    \item Il est moins susceptible de générer des feuilles surchargés ou des nœuds internes 
	avec peu de MBR.

\end{itemize}
Inconvénients du split quadratique :

\begin{itemize}
    \item Il peut être plus coûteux en termes de temps de calcul, car il nécessite l'évaluation 
	de toutes les paires de MBR possibles.
\end{itemize}

\subsubsection {Pick Seed}


Dans le contexte du split quadratique, le pickseed est l'algorithme utilisé pour choisir les 
deux MBR initiaux à diviser.

L'algorithme du pickseed consiste à énumérer toutes les paires possibles de MBR et à sélectionner 
celle qui a l'aire d'encadrement la plus grande.

L'objectif est de trouver les deux MBR qui englobent l'espace maximal, formant ainsi une base solide 
pour la construction ultérieure de l'arbre.

% test
\begin{minted}{java}

protected AbstractNodePair pickSeeds(Node node) {
    AbstractNodePair nodes = new AbstractNodePair(node.getChild(0), 
						    node.getChild(1));
    MBR biggest = nodes.n1.getMBR().getUnion(nodes.n2.getMBR());
    for (int i = 1; i < node.getChildren().size(); i++) {
	for (int j = i + 1; j < node.getChildren().size(); j++) {
	    if (i==j) { continue; }
	    MBR union = node.getChild(i).getMBR().getUnion(node.getChild(j).getMBR());
	    if (union.getArea() > biggest.getArea()) {
		nodes.n1 = node.getChild(i);
		nodes.n2 = node.getChild(j);
		biggest = union;
	    }
	}
    }
    return nodes;
}
\end{minted}
% \end{lstlisting}


\subsubsection {Pick Next}

Après avoir choisi les deux MBR initiaux, le picknext est utilisé pour sélectionner le prochain MBR 
à insérer dans l'arbre lors du processus de construction.

L'algorithme du picknext consiste à évaluer l'augmentation d'aire résultant de l'ajout de chaque MBR 
restant au premier groupe. Le MBR qui entraîne la plus petite augmentation d'aire est choisi 
comme prochain MBR à insérer.

Ce processus est répété jusqu'à ce que tous les MBR soient attribués à un groupe.
% \begin{lstlisting}[language=Java, caption=Algorithme du picknext]
% \begin{minted}[bgcolor=bg]{java}





\subsection {Split linéaire}

Le split linéaire est un autre algorithme de division utilisé dans la construction d'un R-Tree. 
Il ne prend pas en compte la compacité des MBR lors de la sélection des MBR à diviser.
mais se base sur la position des MBR. Ce qui permet de réduire le temps de calcul.

Le split linéaire est une autre variante de l'algorithme de division utilisé dans la construction d'
un R-Tree. 
Contrairement au split quadratique, il ne prend pas en compte la compacité des MBR lors de la 
sélection des MBR à diviser mais se base sur la position relative des MBR les uns par rapport 
aux autres. 
Ce qui permet de réduire le temps de calcul.

L'idée du split linéaire est de sélectionner deux MBR initiaux qui sont le plus extrêmes possible.

Avantages du split linéaire :

\begin{itemize}
    \item Il est plus rapide que le split quadratique, car il ne nécessite pas l'évaluation de
	toutes les paires de MBR possibles.
    \item Il est plus simple à implémenter.

\end{itemize}
Inconvénients du split linéaire :

\begin{itemize}
    \item Il tend à créer une structure moins équilibrée en termes de taille des nœuds et de
	compacité des MBR.
    \item Il est plus susceptible de générer des feuilles surchargés ou des nœuds internes
	avec peu de MBR.

\end{itemize}

\subsubsection {Pick Seed}

Le pickseed (linéaire) est utilisé pour sélectionner les deux MBR initiaux dans le contexte du split linéaire. 
Il s'agit simplement d'attribuer les deux premiers MBR rencontrés à chaque groupe.



\subsubsection {Pick Next}

Le picknext (linéaire) est utilisé pour sélectionner le prochain MBR à insérer dans l'arbre lors de la construction. 
Il suffit de choisir le prochain MBR dans l'ordre de leur apparition lors du parcours.



\section {Recherche}\label{recherche}

\section {Expériences sur donnees réelles}
ici on va ecrire les experiences sur donnees réelles

\subsection {Belgique - Secteurs statistiques}
ici on va ecrire les experiences sur les secteurs statistiques

\subsection {France - Communes}
ici on va ecrire les experiences sur les communes

\subsection {Monde - Pays}
ici on va ecrire les experiences sur les pays

\subsection {Monde - Villes}
ici on va ecrire les experiences sur les villes

\subsection {Analyse}
ici on va ecrire l'analyse

\section {Conclusion}
ici on va ecrire la conclusion

\section {Références bibliographiques}
ici on va ecrire les references bibliographiques




\end {document}
