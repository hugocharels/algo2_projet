\documentclass {article}

\begin {document}
% \begintitlepage
% \centering
\author {Hugo Charels 000544051 \and Mickael Kovel 000396950}
\date {28 Avril 2023}
\title {Rapport de projet : Rectangle Tree}
\maketitle
\newpage


\tableofcontents
\newpage
\section {Introduction}


% Les polygones jouent un rôle essentiel dans de nombreuses applications, 
% qu'il s'agisse de représenter des régions administratives sur une carte, des informations géologiques, 
% des zones dans des dessins vectoriels ou même des imageries médicales. Lors de la manipulation de telles données, 
% une question courante se pose : à quel polygone appartient un point donné, tel qu'un clic de souris ou 
% la position d'un capteur GPS ? Cette question est résolue par le problème du "Point in Polygon" (PIP), 
% dont l'algorithme a été connu depuis 1962.
% \\
Dans de nombreuses applications, la manipulation de polygones est essentielle. Ces polygones peuvent représenter 
des régions administratives, des informations géologiques, des zones de dessins vectoriels ou encore des 
images médicales. Lorsqu'il s'agit de déterminer à quel polygone appartient un point donné, une question se pose : 
comment résoudre efficacement le problème du "Point in Polygon" (PIP) ?

% \\
L'algorithme PIP, connu depuis 1962, consiste à compter le nombre de fois qu'une demi-droite partant du point 
traverse une arête du polygone. Si ce nombre est pair, le point est à l'extérieur du polygone, sinon, 
il est contenu à l'intérieur. Cependant, si l'on a des milliers de polygones complexes à tester avec un 
grand nombre de points, une méthode naïve devient inefficace.

Une première optimisation consiste à utiliser l'enveloppe ou "minimum bounding rectangle" (MBR), 
qui est le plus petit rectangle horizontal englobant totalement le polygone. Avant d'appliquer l'algorithme PIP, 
on vérifie d'abord l'inclusion du point dans le MBR, ce qui permet de réduire le nombre d'appels à PIP.

Cependant, lorsque le nombre de polygones et de MBR est élevé, une approche hiérarchique basée sur l'algorithme R-Tree 
devient pertinente. En regroupant les MBR de manière optimale, on peut éviter de considérer tous les MBR pour 
chaque point à tester.

Dans ce projet, nous allons implémenter et évaluer deux variantes de l'algorithme R-Tree (quadratique et linéaire) 
pour résoudre efficacement le problème du PIP. Nous testerons ces variantes avec différents paramètres pour 
évaluer leurs performances.

Le rapport détaillera la méthodologie utilisée, les résultats obtenus et une analyse approfondie des performances 
des variantes de l'algorithme R-Tree. Nous discuterons également des défis rencontrés et des perspectives pour de 
futures recherches.


En conclusion, ce projet vise à résoudre efficacement le problème du PIP en utilisant l'algorithme R-Tree. 
Les résultats obtenus nous permettront de déterminer la meilleure variante en fonction des paramètres et 
de formuler des recommandations pour d'éventuelles améliorations.

\section {Structure}
ici on va ecrire la structure
\section {Création}
ici on va ecrire la creation
\subsection {Split quadratique}
ici on va ecrire le split quadratique

\subsubsection {Pick Seed}
ici on va ecrire le pick seed

\subsubsection {Pick Next}
ici on va ecrire le pick next

\subsection {Split linéaire}
ici on va ecrire le split linéaire

\subsubsection {Pick Seed}
ici on va ecrire le pick seed

\subsubsection {Pick Next}
ici on va ecrire le pick next

\section {Recherche}
ici on va ecrire la recherche

\section {Expériences sur donnees réelles}
ici on va ecrire les experiences sur donnees réelles

\subsection {Belgique - Secteurs statistiques}
ici on va ecrire les experiences sur les secteurs statistiques

\subsection {France - Communes}
ici on va ecrire les experiences sur les communes

\subsection {Monde - Pays}
ici on va ecrire les experiences sur les pays

\subsection {Monde - Villes}
ici on va ecrire les experiences sur les villes

\subsection {Analyse}
ici on va ecrire l'analyse

\section {Conclusion}
ici on va ecrire la conclusion

\section {Références bibliographiques}
ici on va ecrire les references bibliographiques




\end {document}
